\documentclass{article}
\usepackage{amsmath,amsthm,amssymb}
\usepackage{amsthm}
\usepackage{mathtext}
\usepackage{mathtools}
\usepackage{graphicx}
\usepackage{gensymb}
\usepackage{wrapfig}
\graphicspath{ {images/} }
\usepackage[T1,T2A]{fontenc}
\usepackage[utf8]{inputenc}
\usepackage[english,russian]{babel}
\usepackage{fancyhdr}
\usepackage[a4paper]{geometry}
\pagestyle{fancy}
\fancyhf{}
\fancyhead[L]{Когомологии де Рама.}
\theoremstyle{definition}
\newtheorem{defn}{Def}
\newtheorem{example}{Ex}
 \newtheorem{theorem}{Th}
 \DeclareMathOperator{\im}{im}
\begin{document}
	[Савченко 5.5.5, 19.01]\textit{Баллон вместимостью 50 литров наполнили воздухом при 27\degree С до давления 10 МПа. Какой объем воды можно вытеснить из цистерны подводной лодки воздухом этого баллона, если вытеснение производится на глубине 40 метров? Температура воздуха после расширения 3\degree С.}
	
	Уравнение состояния для воздуха в баллоне до выпускания воздуха:
	$$P V = \nu RT$$
	После выпускания части воздуха:
	$$P_1 V = \nu_1 RT$$
	В цистерне с водой:
	$$P_2 V_{выт} = \nu_2 R T_{к}$$
	где $\nu_1 + \nu_2 = \nu$. 
	Условие равновесия: $$P_1 = P_2 = P_{атм} + \rho_{в} g h$$
	Выражая $V_{выт}$:
	$$V_{выт} = V \frac{\nu_2 T_к}{\nu_1 T} = V (\frac{\nu}{\nu_1} - 1) \frac{T_к}{T}$$
	\[
	V_{выт}=\boxed{V\left( \frac{P}{P_{атм}+\rho_в g h} - 1 \right) \frac{T_к}{T}}
	\]
	Численный ответ - $874$ литра.
	
\end{document}
