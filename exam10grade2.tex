\documentclass[a5paper, landscape]{exam}
\usepackage{amsmath}
\usepackage{amsmath,amsthm,amssymb}
\usepackage{mathtext}
\usepackage{mathtools}
\usepackage[T1,T2A]{fontenc}
\usepackage[utf8x]{inputenc}
\usepackage[english,russian]{babel}
\DeclareMathOperator{\Tr}{Tr}
\DeclareMathOperator{\ad}{ad}
\begin{document}
		\firstpageheader{Физика}{}{10 класс}
			\firstpageheadrule

\centering{Вариант 2}
\vspace*{0.15cm}
\begin{questions}
\question	Сформулируйте и докажите теорему Гаусса. Дайте определение потенциала. Получите формулы напряженности и потенциала равномерно заряженной сферы.
\question Обратимые и необратимые процессы. Второе начало термодинамики. Цикл Карно в естественных координатах (S-T). Сформулируйте и докажите первую теорему Карно (КПД цикла Карно). Сформулируйте и докажите вторую теорему Карно(Максимальный КПД тепловой машины).
\question На шероховатой горизонтальной поверхности стола покоится чаша. Внутренняя поверхность чаши - гладкая полусфера радиуса $R$. На дне чаши лежит небольшая шайба массы $m$. Масса чаши $3m$. Ударом шайбе сообщают горизонтальную скорость $V_0 = \sqrt{2gR}$. Скольжение чаши начинается в тот момент времени, когда вектор скорости шайбы повернется на угол $\alpha = \pi/6$. С какой силой P шайба действует на чашу в этот момент? Вычислите коэффицент $\mu$ трения скольжения чаши по столу.
\question Задан процесс в координатах $P-V$. Вычислите КПД.
\question Вольт-амперная характеристика лампы накаливания описывается законом $I=k_1 \sqrt{U_l}$, диода - законом $I = k_2 U^2_d$. Лампа и диод обладают следующим свойством: если подключить любой из элементов к источнику, то мощность тепловых потерь будет максимальной для данного источника. Если подключить диод и лампу последовательно то мощность потерь будет равна $P_1$. Какова будет мощность при последовательном подключении?
\end{questions}
\end{document}