\documentclass[a4paper]{exam}
\usepackage{amsmath}
\usepackage{amsmath,amsthm,amssymb}
\usepackage{mathtext}
\usepackage{mathtools}
\usepackage[left=2cm,right=2cm,
top=2cm,bottom=1cm,bindingoffset=0cm]{geometry}
\usepackage[T1,T2A]{fontenc}
\usepackage{bbding}
\usepackage{lipsum}
\usepackage{tikz}
\usepackage{wrapfig}
\usepackage{graphics}
\usepackage[utf8x]{inputenc}
\usepackage[english]{babel}
\graphicspath{{.}}
\begin{document}
\firstpageheader{Физика}{}{Осень 2019}
\firstpageheadrule
\begin{center}
	Теоретические вопросы к экзамену по физике
\end{center}
\begin{questions}
\question Магнитное поле. Действие магнитного поля на рамку с током. Магнитная индукция. Закон Био-Савара-Лапласа. Примеры.
\question Сила, действующая на проводник с током в магнитном поле. Закон Ампера. Действие магнитного поля на движущийся заряд. Сила Лоренца.
\question Магнитный поток. Опыты Фарадея. Явление электромагнитной индукции. Принцип работы трансформатора. Вихревое электрическое поле. Закон электромагнитной индукции. Правило Ленца.
\question Теорема о циркуляции вектора индукции магнитного и электрического поля. Условие вихреобразности поля.
\question Уравнения Максвелла как обобщение экспериментальных данных. Ток смещения. Вихревое электрическое поле. Взаимные превращения электрического и магнитного полей..
\question Вектор Умова-Пойнтинга. Закон сохранения энергии электромагнитного поля.

 

\question  Методы комплексных амплитуд и векторных диаграмм. Активное, емкостное и индуктивное сопротивление. Закон Ома для цепей переменного тока.
\question Колебательный контур. Собственные колебания в контуре. Уравнение гармонических колебаний. Энергия, запасенная в контуре. 
\question Затухающие колебание в $RLC$ контуре. 
\question Вынужденные колебания. Резонанс.

 


\question Опытные законы оптики. Абсолютный и относительный показатели преломления. Полное внутреннее отражение. 
\question Принцип Ферма. Основные законы распространения света.
\question Закон отражения, преломления света и их связь с принципом Ферми. Абсолютный и относительный показатели преломления. Ход лучей в призме. Явление полного отражения.
\question Тонкие линзы. Построение изображений в тонких линзах. Формула тонкой линзы.
\question Электромагнитные волны. Волновое уравнение. Скорость распространения. Поперечность электромагнитных волн. 
\question Интерференция света. Когерентность. 
\question Интерференционная картина от двух разнесенных когерентных источников.
\question   Условие минимума и максимума при интерференции двух плоских волн.
 


\end{questions}




\end{document}