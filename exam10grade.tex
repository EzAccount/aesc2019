\documentclass[a5paper, landscape]{exam}
\usepackage{amsmath}
\usepackage{amsmath,amsthm,amssymb}
\usepackage{mathtext}
\usepackage{mathtools}
\usepackage[T1,T2A]{fontenc}
\usepackage[utf8x]{inputenc}
\usepackage[english,russian]{babel}
\DeclareMathOperator{\Tr}{Tr}
\DeclareMathOperator{\ad}{ad}
\begin{document}
		\firstpageheader{Физика}{}{10 класс}
			\firstpageheadrule
\centering{Вариант 1}
\vspace*{0.15cm}
\begin{questions}
\question	Сформулируйте теорему Гаусса. Получите закон Кулона из теоремы Гаусса. Получите формулы напряженности бесконечной заряженной плоскости, шара, сферы.
\question Дайте определение теплоемкости, политропического процесса. Выведите уравнения политропического процесса. Рассмотрите частные случаи политропического процесса. Выведите уравнение Майера.
\question На шероховатой горизонтальной поверхности стола покоится чаша. Внутренняя поверхность чаши - гладкая полусфера радиуса $R$. На дне чаши лежит небольшая шайба массы $m$. Масса чаши $3m$. Ударом шайбе сообщают скорость $V_0 = \sqrt{2gR}$. Сколжение чаши начинается в тот момент времени, когда вектор скорости шайбы повернется на угол $\alpha = \pi/6$. С какой силой P шайба действует на чашу в этот момент? Вычислите коэффицент $\mu$ трения скольжения чаши по столу.
\question Задан график циклического процесса произведенного над неизвестным количеством газа. Известно, что $C_d = 1000 $Дж/К, $C_a = 0.715$Дж/К, а так же $T_c - T_b = 2 (T_b - T_a)=200$К и $\frac{p_c}{p_a}=\frac{V_c}{V_a}$ Найдите работу газа за цикл и КПД цикла $\eta$.
\question Два металлических одинаковых полушара радиуса $R$ расположены так, что между ними имеется очень небольшой зазор. Полушары заряжают зарядами $-Q$ и $3Q$. Найти напряженность электрического поля в зазоре между полушарами.
\end{questions}
\end{document}