\documentclass[a5paper, landscape]{exam}
\usepackage{amsmath}
\usepackage{amsmath,amsthm,amssymb}
\usepackage{mathtext}
\usepackage{mathtools}
\usepackage[left=2cm,right=2cm,
top=2cm,bottom=1cm,bindingoffset=0cm]{geometry}
\usepackage[T1,T2A]{fontenc}
\usepackage{bbding}
\usepackage{lipsum}
\usepackage{tikz}
\usepackage{wrapfig}
\usepackage{graphics}
\usepackage[utf8x]{inputenc}
\usepackage[english]{babel}
\graphicspath{{.}}
\begin{document}
\rule{1\textwidth}{0.4pt}
\center{Билет 1}
	\begin{questions}
	\question Магнитное поле. Действие магнитного поля на рамку с током. Магнитная индукция. Закон Био-Савара-Лапласа. Примеры.
	\question Принцип Ферми. Основные законы распространения света.
	\question Два удалённых проводящих шара радиусом $R$ соединены участком цепи, содержащим конденсатор ёмкостью $C$, катушку индуктивностью $L$ и ключ $K$(см.рисунок). В начальный момент времени конденсатор заряжендо напряжения $U_0$, заряды на шарах отсутствуют. Определите максимальный заряд на шаре после замыкания ключа $K$.
	
	\includegraphics[width=0.2\textwidth]{exam_task_1}


	\question Проводник движется равноускоренно в однородном вертикальном магнитном поле. Направление скорости перпендикулярно проводнику. Длина проводника — 2 м. Индукция перпендикулярна проводнику и скорости его движения. Проводник перемещается на 3 м за некоторое время. При этом начальная скорость проводника равна нулю, а ускорение 5 м/с2. Найдите индукцию магнитного поля, зная, что ЭДС индукции на концах проводника в конце движения равна 2 В.
\end{questions}


\newpage
\rule{1\textwidth}{0.4pt}
\center{Билет 2}
\begin{questions}
\question Сила, действующая на проводник с током в магнитном поле. Закон Ампера. Действие магнитного поля на движущийся заряд. Сила Лоренца.
\question Закон преломления света и его связь с принципом Ферми. Абсолютный и относительный показатели преломления. Ход лучей в призме. Явление полного отражения.
\question Два удалённых проводящих шара радиусом $R$ соединены участком цепи, содержащим конденсатор ёмкостью $C$, катушку индуктивностью $L$ и ключ $K$(см.рисунок). В начальный момент времени конденсатор заряжендо напряжения $U_0$, заряды на шарах отсутствуют. Определите максимальный заряд на шаре после замыкания ключа $K$.

\includegraphics[width=0.2\textwidth]{exam_task_1}


\question Проводник движется равноускоренно в однородном вертикальном магнитном поле. Направление скорости перпендикулярно проводнику. Длина проводника — 2 м. Индукция перпендикулярна проводнику и скорости его движения. Проводник перемещается на 3 м за некоторое время. При этом начальная скорость проводника равна нулю, а ускорение 5 м/с2. Найдите индукцию магнитного поля, зная, что ЭДС индукции на концах проводника в конце движения равна 2 В.
\end{questions}

\vspace{\fill}
\newpage

\rule{1\textwidth}{0.4pt}
\center{Билет 3}
\begin{questions}
\question Магнитный поток. Опыты Фарадея. Явление электромагнитной индукции. Вихревое электрическое поле. Закон электромагнитной индукции. Правило Ленца.
\question Тонкие линзы. Построение изображений в тонких линзах.
\question Два удалённых проводящих шара радиусом $R$ соединены участком цепи, содержащим конденсатор ёмкостью $C$, катушку индуктивностью $L$ и ключ $K$(см.рисунок). В начальный момент времени конденсатор заряжендо напряжения $U_0$, заряды на шарах отсутствуют. Определите максимальный заряд на шаре после замыкания ключа $K$.

\includegraphics[width=0.2\textwidth]{exam_task_1}


\question Проводник движется равноускоренно в однородном вертикальном магнитном поле. Направление скорости перпендикулярно проводнику. Длина проводника — 2 м. Индукция перпендикулярна проводнику и скорости его движения. Проводник перемещается на 3 м за некоторое время. При этом начальная скорость проводника равна нулю, а ускорение 5 м/с2. Найдите индукцию магнитного поля, зная, что ЭДС индукции на концах проводника в конце движения равна 2 В.
\end{questions}


\newpage

\rule{1\textwidth}{0.4pt}
\center{Билет 4}
\begin{questions}
\question Вынужденные колебания. Резонанс.
\question Интерференция света. Когерентность. 
\question Два удалённых проводящих шара радиусом $R$ соединены участком цепи, содержащим конденсатор ёмкостью $C$, катушку индуктивностью $L$ и ключ $K$(см.рисунок). В начальный момент времени конденсатор заряжендо напряжения $U_0$, заряды на шарах отсутствуют. Определите максимальный заряд на шаре после замыкания ключа $K$.

\includegraphics[width=0.2\textwidth]{exam_task_1}


\question Проводник движется равноускоренно в однородном вертикальном магнитном поле. Направление скорости перпендикулярно проводнику. Длина проводника — 2 м. Индукция перпендикулярна проводнику и скорости его движения. Проводник перемещается на 3 м за некоторое время. При этом начальная скорость проводника равна нулю, а ускорение 5 м/с2. Найдите индукцию магнитного поля, зная, что ЭДС индукции на концах проводника в конце движения равна 2 В.
\end{questions}


\newpage
\rule{1\textwidth}{0.4pt}
\center{Билет 5}
\begin{questions}
\question Уравнения Максвелла как обобщение экспериментальных данных. 
\question Интерференционная картина от двух разнесенных когерентных источников.
\question Два удалённых проводящих шара радиусом $R$ соединены участком цепи, содержащим конденсатор ёмкостью $C$, катушку индуктивностью $L$ и ключ $K$(см.рисунок). В начальный момент времени конденсатор заряжендо напряжения $U_0$, заряды на шарах отсутствуют. Определите максимальный заряд на шаре после замыкания ключа $K$.

\includegraphics[width=0.2\textwidth]{exam_task_1}


\question Проводник движется равноускоренно в однородном вертикальном магнитном поле. Направление скорости перпендикулярно проводнику. Длина проводника — 2 м. Индукция перпендикулярна проводнику и скорости его движения. Проводник перемещается на 3 м за некоторое время. При этом начальная скорость проводника равна нулю, а ускорение 5 м/с2. Найдите индукцию магнитного поля, зная, что ЭДС индукции на концах проводника в конце движения равна 2 В.
\end{questions}


\newpage

\rule{1\textwidth}{0.4pt}
\center{Билет 6}
\begin{questions}
\question Вектор Умова-Пойнтинга. Закон сохранения энергии электромагнитного поля.
\question Формула тонкой линзы.
\question Два удалённых проводящих шара радиусом $R$ соединены участком цепи, содержащим конденсатор ёмкостью $C$, катушку индуктивностью $L$ и ключ $K$(см.рисунок). В начальный момент времени конденсатор заряжендо напряжения $U_0$, заряды на шарах отсутствуют. Определите максимальный заряд на шаре после замыкания ключа $K$.

\includegraphics[width=0.2\textwidth]{exam_task_1}


\question Проводник движется равноускоренно в однородном вертикальном магнитном поле. Направление скорости перпендикулярно проводнику. Длина проводника — 2 м. Индукция перпендикулярна проводнику и скорости его движения. Проводник перемещается на 3 м за некоторое время. При этом начальная скорость проводника равна нулю, а ускорение 5 м/с2. Найдите индукцию магнитного поля, зная, что ЭДС индукции на концах проводника в конце движения равна 2 В.
\end{questions}


\newpage

\rule{1\textwidth}{0.4pt}
\center{Билет 7}
\begin{questions}
\question Теорема о циркуляции вектора индукции магнитного поля. Примеры.
\question Электромагнитные волны. Скорость их распространения. Поперечность электромагнитных волн. 
\question Два удалённых проводящих шара радиусом $R$ соединены участком цепи, содержащим конденсатор ёмкостью $C$, катушку индуктивностью $L$ и ключ $K$(см.рисунок). В начальный момент времени конденсатор заряжендо напряжения $U_0$, заряды на шарах отсутствуют. Определите максимальный заряд на шаре после замыкания ключа $K$.

\includegraphics[width=0.2\textwidth]{exam_task_1}


\question Проводник движется равноускоренно в однородном вертикальном магнитном поле. Направление скорости перпендикулярно проводнику. Длина проводника — 2 м. Индукция перпендикулярна проводнику и скорости его движения. Проводник перемещается на 3 м за некоторое время. При этом начальная скорость проводника равна нулю, а ускорение 5 м/с2. Найдите индукцию магнитного поля, зная, что ЭДС индукции на концах проводника в конце движения равна 2 В.
\end{questions}


\newpage

\rule{1\textwidth}{0.4pt}
\center{Билет 8}
\begin{questions}
\question Закон отражения и преломления света. Связь с принципом Ферми.
\question  Методы комплексных амплитуд и векторных диаграмм. Активное, емкостное и индуктивное сопротивление. Закон Ома для цепей переменного тока.
 
\question Два удалённых проводящих шара радиусом $R$ соединены участком цепи, содержащим конденсатор ёмкостью $C$, катушку индуктивностью $L$ и ключ $K$(см.рисунок). В начальный момент времени конденсатор заряжендо напряжения $U_0$, заряды на шарах отсутствуют. Определите максимальный заряд на шаре после замыкания ключа $K$.

\includegraphics[width=0.2\textwidth]{exam_task_1}


\question Проводник движется равноускоренно в однородном вертикальном магнитном поле. Направление скорости перпендикулярно проводнику. Длина проводника — 2 м. Индукция перпендикулярна проводнику и скорости его движения. Проводник перемещается на 3 м за некоторое время. При этом начальная скорость проводника равна нулю, а ускорение 5 м/с2. Найдите индукцию магнитного поля, зная, что ЭДС индукции на концах проводника в конце движения равна 2 В.
\end{questions}

\newpage

\rule{1\textwidth}{0.4pt}
\center{Билет 9}
\begin{questions}
\question Затухающие колебание в $RLC$ контуре. 
\question   Условие минимума и максимума при интерференции двух плоских волн.
 
\question Два удалённых проводящих шара радиусом $R$ соединены участком цепи, содержащим конденсатор ёмкостью $C$, катушку индуктивностью $L$ и ключ $K$(см.рисунок). В начальный момент времени конденсатор заряжендо напряжения $U_0$, заряды на шарах отсутствуют. Определите максимальный заряд на шаре после замыкания ключа $K$.

\includegraphics[width=0.2\textwidth]{exam_task_1}


\question Проводник движется равноускоренно в однородном вертикальном магнитном поле. Направление скорости перпендикулярно проводнику. Длина проводника — 2 м. Индукция перпендикулярна проводнику и скорости его движения. Проводник перемещается на 3 м за некоторое время. При этом начальная скорость проводника равна нулю, а ускорение 5 м/с2. Найдите индукцию магнитного поля, зная, что ЭДС индукции на концах проводника в конце движения равна 2 В.
\end{questions}

\newpage

\rule{1\textwidth}{0.4pt}
\center{Билет 10}
\begin{questions}
\question Колебательный контур. Собственные колебания в контуре. Уравнение гармонических колебаний. Энергия, запасенная в контуре. 
\question Уравнения Максвелла как обобщение экспериментальных данных. Ток смещения. Вихревое электрическое поле. Взаимные превращения электрического и магнитного полей.
 
\question Два удалённых проводящих шара радиусом $R$ соединены участком цепи, содержащим конденсатор ёмкостью $C$, катушку индуктивностью $L$ и ключ $K$(см.рисунок). В начальный момент времени конденсатор заряжендо напряжения $U_0$, заряды на шарах отсутствуют. Определите максимальный заряд на шаре после замыкания ключа $K$.

\includegraphics[width=0.2\textwidth]{exam_task_1}


\question Проводник движется равноускоренно в однородном вертикальном магнитном поле. Направление скорости перпендикулярно проводнику. Длина проводника — 2 м. Индукция перпендикулярна проводнику и скорости его движения. Проводник перемещается на 3 м за некоторое время. При этом начальная скорость проводника равна нулю, а ускорение 5 м/с2. Найдите индукцию магнитного поля, зная, что ЭДС индукции на концах проводника в конце движения равна 2 В.
\end{questions}
\newpage

\rule{1\textwidth}{0.4pt}
\center{Билет 11}
\begin{questions}
\question Электромагнитная индукция. Принцип работы трансформатора.
\question Интерференция света. Когерентность.
\question Два удалённых проводящих шара радиусом $R$ соединены участком цепи, содержащим конденсатор ёмкостью $C$, катушку индуктивностью $L$ и ключ $K$(см.рисунок). В начальный момент времени конденсатор заряжендо напряжения $U_0$, заряды на шарах отсутствуют. Определите максимальный заряд на шаре после замыкания ключа $K$.

\includegraphics[width=0.2\textwidth]{exam_task_1}


\question Проводник движется равноускоренно в однородном вертикальном магнитном поле. Направление скорости перпендикулярно проводнику. Длина проводника — 2 м. Индукция перпендикулярна проводнику и скорости его движения. Проводник перемещается на 3 м за некоторое время. При этом начальная скорость проводника равна нулю, а ускорение 5 м/с2. Найдите индукцию магнитного поля, зная, что ЭДС индукции на концах проводника в конце движения равна 2 В.
\end{questions}

\newpage
\rule{1\textwidth}{0.4pt}
\center{Билет 12}
\begin{questions}
\question Теорема о циркуляции вектора индукции электромагнитного поля. Условие вихреобразности поля.
\question Опытные законы оптики. Абсолютный и относительный показатели преломления. Полное внутреннее отражение. 
\question Два удалённых проводящих шара радиусом $R$ соединены участком цепи, содержащим конденсатор ёмкостью $C$, катушку индуктивностью $L$ и ключ $K$(см.рисунок). В начальный момент времени конденсатор заряжендо напряжения $U_0$, заряды на шарах отсутствуют. Определите максимальный заряд на шаре после замыкания ключа $K$.

\includegraphics[width=0.2\textwidth]{exam_task_1}


\question Проводник движется равноускоренно в однородном вертикальном магнитном поле. Направление скорости перпендикулярно проводнику. Длина проводника — 2 м. Индукция перпендикулярна проводнику и скорости его движения. Проводник перемещается на 3 м за некоторое время. При этом начальная скорость проводника равна нулю, а ускорение 5 м/с2. Найдите индукцию магнитного поля, зная, что ЭДС индукции на концах проводника в конце движения равна 2 В.
\end{questions}
\end{document}