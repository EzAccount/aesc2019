\documentclass[12pt,a5paper]{article}
\usepackage{geometry}
\geometry{
	a5paper,
	left=10mm,
	right=15mm,
	bottom=10mm,
	top=15mm,
}
\usepackage[utf8]{inputenc}
\usepackage[T2A]{fontenc}
\usepackage{amsmath}
\usepackage{amsthm}
\usepackage{amssymb}
\usepackage{mathtools}
\usepackage{amsfonts}
 \pagenumbering{gobble}
\begin{document}
\noindent
\begin{enumerate}
	\item По тонкому кольцу радиуса $R$ распределён заряд $q$. Найдите потенциал поля этого заряда в центре кольца.
	\item В вершинах правильного треугольника расположены точечные заряды $q$, $2q$ и $3q$. Найдите потенциал.
	\item Две концентрические сферы радиуса $R$ и $2R$ равномерно заряжены по поверхности зарядами $2q$ и $-2q$ соответственно. Найдите потенциал электрического поля во всем пространстве.
	\item Две тонкие бесконечные параллельные пластины расположены на расстоянии $d$ друг от друга. По поверхности пластин равномерно распределены заряды с поверхностной плотностью $\sigma$ и $\sigma_1$. Найдите потенциал электрического поля во всем пространстве, если $1)\ \sigma = \sigma_1$, $2)\ \sigma_1 = -\sigma$, $3)\ \sigma_1 = 2 \sigma$
\end{enumerate}
\end{document}